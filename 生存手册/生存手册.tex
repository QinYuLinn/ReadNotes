\input{/Users/qinyulin/LaTex/Modules/ArticleClass/preamble}
\input{/Users/qinyulin/LaTex/Modules/ArticleClass/contents}
\input{/Users/qinyulin/LaTex/Modules/ArticleClass/mathequ}
\input{/Users/qinyulin/LaTex/Modules/ArticleClass/environments}
\input{/Users/qinyulin/LaTex/Modules/ArticleClass/tables}
\input{/Users/qinyulin/LaTex/Modules/ArticleClass/figures}
\input{/Users/qinyulin/LaTex/Modules/ArticleClass/float}
\input{/Users/qinyulin/LaTex/Modules/ArticleClass/colors}
\input{/Users/qinyulin/LaTex/Modules/ArticleClass/fancypage}
\setlength{\droptitle}{-2cm}
\pretitle{\begin{center}\LARGE\sffamily}
\title{生存手册阅读笔记}
\posttitle{\par\end{center}\vspace{-0.3cm}}
\preauthor{\large}
\DeclareRobustCommand{\authorthing}
{
\begin{center}
\begin{tabular}{c}%cc}
覃宇林\\
\end{tabular}
\end{center}
}
\author{\authorthing}
\postauthor{}
\predate{\begin{center}\large\scshape}
\date{2021年11月29日}
\postdate{\par\end{center}}

\begin{document}
%pagestyle{plain}
\frontpagestyle
\maketitle
\pagenumbering{Roman}
\tableofcontents\newpage
\pagenumbering{arabic}
\mainpagestyle
\setcounter{page}{1}
\section{必须物品}
求生宝盒应装物品:火柴,蜡烛,打火石,放大镜,针线,鱼钩鱼线,指南针,$\beta$灯,圈套索线,弹性锯条,医疗小瓶(镇痛药,肠道镇定剂,抗生素,抗组胺药类,漂白粉,抗疟疾;类药品,高锰酸钾),外科手术刀片,蝴蝶结,膏药类,避孕套。\par
救生箱:铝制饭盒,凝固态燃料块,手电筒,闪光信号灯,标记板,火柴,茶叶,食品,救生袋。\par
挑选合适的刀。\par
基本需要:水,火,庇护所,食物。首要需求因环境变化,首先确定首要需求。\par
流动水最好。严重缺水时不要进食或者尽可能少进食能降低身体水分消耗,严重缺水时可以用鼻子呼吸代替嘴呼吸降低身体水分消耗。寻找水源在山谷底部或挖掘绿色植物分布带。干涸河床或沟渠下面可能有泉眼。高山地区寻水要沿着岩石缝隙寻找。海岸边在最高水线以上挖坑寻水。岩石的断层间也可能有水。\par
可以采集雨露当作饮用水。长时间缺水后不能豪饮。\par
可以根据动物来寻找水源,尤其是食草性动物,肉食性动物不一定。谷食性鸟类也是找水的信号灯。昆虫类也是找水的信号灯。\par
在一段树木的嫩枝上套上塑料袋,页面蒸腾作用会在袋内产生凝结水。将塑料薄膜覆盖在生长良好的植物上也可以收集凝结水。将刚砍断的新鲜植物的枝叶放在大塑料袋里,温度升高时也会产生凝结水。\par
日光蒸馏法也可以用来收集水。无论何时都不要饮用海水和尿液,但经过蒸馏,两者可以产生饮用水。\par
冰雪化水可以饮用,海上古老的冰块(边缘不那么光滑,一般呈天蓝色)化水含盐低,可以饮用。\par
也可以从植被中取水,如杯型植物等(叶片长呈中空状,经常蓄有水),竹类等。\par
汁液有毒植物的径被砍断时会产生浓的乳白色汁液。\par
也可以从动物的流汁里取水,如鱼等等。\par





\end{document}