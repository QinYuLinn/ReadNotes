the lagrangian:
\[\mathcal{L}=\bar{\Psi}(i\gamma^\mu\partial_\mu-m)\Psi\]
\subsection{representation of the lorentz group espically for 4 dimensions}
if we define 
\[J^{\mu,\nu}=i(x^{\mu}\partial^{\nu}-x^{\nu}\partial^{\mu})\]
then the six operator generate the three boost and three rotation of the lorentz group.\par
\[[J^{\mu,\nu},J^{\rho,\sigma}]=i(g^{\nu \rho}J^{\mu \sigma}-g^{\mu \rho}J^{\nu \sigma}-g^{\nu \sigma}J^{\mu \rho}+g^{\mu \sigma}J^{\nu \rho})\]
to clearly see theis operators is the generator ,we can use them to form the lorentz transfer:
\[V\rightarrow V'=(I-\frac{iJ^{\mu\nu}}{2}w_{\mu,\nu})V\]
in the above description, the $w_{\mu,\nu}$is just elements a random metric w which describ a lorentz tranferomation.\par
dirac's trick for the representation of lorentz group for n dimension:\par
if $\gamma^{\mu}$ is the n dimension metrics satisfying the relation:
\[\{\gamma^{\mu},\gamma^{\nu}\}=2g^{\mu\nu}I\]
then the six metrics:
\[S^{\mu\nu}=\frac{i}{4}[\gamma^{\mu},\gamma^{\nu}]\]
is the generator of the lorentz group for n dimensional representation(to prove  this, we just need to show the commutation relations ).\par

\subsection{the dirac algebra}

the dirac $\gamma$ metrics:
\begin{equation}
\sigma^1=\left(
\begin{array}{cc}
0&1\\
1&0\\
\end{array}
\right)
\sigma^2=\left(
\begin{array}{cc}
0&-i\\
i&0\\
\end{array}
\right)
\sigma^3=\left(
\begin{array}{cc}
1&0\\
0&-1\\
\end{array}
\right)
\end{equation}


\begin{equation}
\gamma^0=\left(
\begin{array}{cc}
0&1\\
1&0\\
\end{array}
\right)
\gamma^i=\left(
\begin{array}{cc}
0&\sigma^i\\
-\sigma^i&0\\
\end{array}
\right)
\end{equation}

then use the dirac's trick and we get the generator of the lorentz group:
\begin{equation}
S^{0i}=-\frac{i}{2}\left(
\begin{array}{cc}
\sigma^i&0\\
0&-\sigma^i\\
\end{array}
\right)
S^{ij}=\frac{1}{2}\epsilon^{ijk}\left(
\begin{array}{cc}
\sigma^k&0\\
0&\sigma^k\\
\end{array}
\right)=\frac{1}{2}\epsilon^{ijk}\Sigma^k
\end{equation}

some properties of the generator:
\[[\gamma^{\mu},S^{\rho,\sigma}]=(J^{\rho\sigma})^\mu_\nu\gamma^\nu\]
\[\Lambda_{\frac{1}{2}}^{-1}\gamma^\mu\Lambda_{\frac{1}{2}}=\Lambda^\mu_\nu \gamma^\nu\]
\[\Lambda_{\frac{1}{2}}=exp({-\frac{i}{2}\omega_{\mu\nu}S^{\mu\nu}})\]
since the metrics $S^{\mu\nu}$ is not hermitian, so we should take care of it when related to corresponding calculations.\par
some properties of these metrics:
\[\sigma^2\vec{\sigma}^*=-\vec{\sigma}\sigma^2\]
\[(p\bullet\sigma)(p\bullet\bar{\sigma})=p^2\]
we define 4 vector :
\[\sigma^\mu=(1,\vec{\sigma}),\bar{\sigma}^\mu=(1,-\vec{\sigma})\]
then the gamma metrics have a unit form:
\begin{equation}
\gamma^\mu=\left(
\begin{array}{cc}
0&\sigma^\mu\\
\bar{\sigma}^\mu&0\\
\end{array}
\right)
\end{equation}
we can use sixteen constant metrics to form a basis for the 4-dimensional metrics space:
\[1,\gamma^\mu,\gamma^{\mu\nu}=\gamma^{[\mu}\gamma^{\nu]},\gamma^{[\mu}\gamma^{\nu}\gamma^{\rho]},\gamma^{[\mu}\gamma^{\nu}\gamma^{\rho}\gamma^{\sigma]}\]
and we can use $\gamma^5$to simply the expresion for the last 5 metrics:
\[\gamma^5=i\gamma^0\gamma^1\gamma^2\gamma^3=-\frac{i}{4!}\epsilon^{\mu\nu\rho\sigma}\gamma_\mu\gamma_\nu\gamma_\rho\gamma_\sigma\]
we can the clrearly see that:
\[\gamma^{[\mu}\gamma^{\nu}\gamma^{\rho}\gamma^{\sigma]}=-i\epsilon^{\mu\nu\rho\sigma}\gamma^5\]
\[\gamma^{[\mu}\gamma^{\nu}\gamma^{\rho]}=-i\epsilon^{\mu\nu\rho\sigma}\gamma_\sigma\gamma^5\]
the properties of the $gamma^5$:
\[(\gamma^5)^\dagger=\gamma^5\]
\[(\gamma^5)^2=1\]
\[\{\gamma^5,\gamma^\mu\}=0\]
\[[\gamma^5,S^{\mu\nu}]=0\]
in dirac's representaion, we have:
\begin{equation}
\gamma^5=\left(
\begin{array}{cc}
-1&0\\
0&1\\
\end{array}
\right)
\end{equation}
the standard choice of theses metrics:
\[1,\gamma^\mu,\sigma^{\mu\nu}=\frac{i}{2}[\gamma^\mu,\gamma^\nu],\gamma^\mu\gamma^5,\gamma^5\]
a property of the Pauli metrics:
\[(\sigma^\mu)_{\alpha\beta}(\sigma_\mu)_{\gamma\delta}=2\epsilon_{\alpha\gamma}\epsilon_{\beta\delta}\]
\[(\bar{\sigma}^\mu)_{\alpha\beta}(\bar{\sigma}_\mu)_{\gamma\delta}=2\epsilon_{\alpha\gamma}\epsilon_{\beta\delta}\]

\subsection{classic solution to dirac equation }
the weyl spinor:
\[i\bar{\sigma}\partial \Psi_L=0\]
\[i\sigma\partial\Psi_R=0\]
the solution to the dirac equation:
\[(i\gamma^{\mu}\partial_\mu-m)\Psi(x)=0\]
using fourier tranfer we get the solotion for positive frequency:
\[\Psi(x)=\int \frac{d^4p}{(2\pi)^4)}u(p)e^{-ipx}\rightarrow (p_\mu\gamma^\mu-m)u(p)=0\]
the solution is :
\begin{equation}
u^s(p)=\left(
\begin{array}{c}
\sqrt{p\bullet\sigma}\xi^s\\
\sqrt{p\bullet\bar{\sigma}}\xi^s\\
\end{array}
\right)
\end{equation}
and the normalization is :
\[\bar{u}^ru^s=2m\delta^{r,s}\rightarrow (u^r)^{\dagger}u^s=2E_p\delta^{r,s}\]
the helicity operator:
\[\hat{h}=\hat{p}\bullet S=\frac{1}{2}\hat{p_i}\left(
\begin{array}{cc}
\sigma^i&0\\
0&\sigma^i\\
\end{array}
\right)
\]
using fourier tranfer we get the solotion for negetive frequency:
\[\Psi(x)=\int \frac{d^4p}{(2\pi)^4)}v(p)e^{ipx}\rightarrow (p_\mu\gamma^\mu+m)v(p)=0\]
the solution is :
\begin{equation}
v^s(p)=\left(
\begin{array}{c}
\sqrt{p\bullet\sigma}\eta^s\\
-\sqrt{p\bullet\bar{\sigma}}\eta^s\\
\end{array}
\right)
\end{equation}
and the normalization is :
\[\bar{v}^rv^s=-2m\delta^{r,s}\rightarrow (v^r)^{\dagger}v^s=2E_p\delta^{r,s}\]
\[\sum_s \bar{u}^s(p)u^s(p)=\gamma\bullet p+m\]
\[\sum_s \bar{v}^s(p)v^s(p)=\gamma\bullet p-m\]

\subsection{quantilization of the dirac field}
\[\mathcal{L}=\bar{\Psi}(i\gamma^\mu\partial_\mu-m)\Psi\]
\[H=\int d^3x\bar{\Psi}(-i\gamma\bullet \nabla+m)\Psi=\int d^3x\Psi^\dagger(-i\gamma_0\gamma\bullet \nabla+m\gamma_0)\Psi\]
the quantilized dirac field is:
\[\Psi(x)=\int \frac{d^3p}{(2\pi)^3}\frac{1}{\sqrt{2E_p}}\sum_s(a_p^su^s(p)e^{-ipx}+(b_p^s)^\dagger v^s(p)e^{ipx})\]
the anticommutation relations are:
\[\{a_p^r,(a_q^s)^\dagger\}=\{b_p^r,(b_q^s)^\dagger\}=(2\pi)^3\delta^3(p-q)\delta^{rs}\]
\[\{\Psi_a(x),\Psi_b^\dagger(y)\}=\delta^3(x-y)\delta_{ab}\]
then the hamitonian is :
\[H=\int \frac{d^3p}{(2\pi)^3}\sum_sE_p((a_p^s)^\dagger a_p^s+(b_p^s)^\dagger b_p^s)\]
the total monmentum operator is:
\[P=\int \frac{d^3p}{(2\pi)^3}\sum_s p((a_p^s)^\dagger a_p^s+(b_p^s)^\dagger b_p^s)\]
the angle monmentum is :
\[J=\int d^3x\Psi^\dagger(\vec{x}\times(-i\nabla)+\frac{1}{2}\Sigma)\Psi\]
the total charge:
\[Q=\int \frac{d^3p}{(2\pi)^3}\sum_s ((a_p^s)^\dagger a_p^s-(b_p^s)^\dagger b_p^s)\]
\subsubsection{the feyman propagator for dirac field}
the retared green function:
\[S_R(x-y)=(i\slashed{\partial}_x+m)D_R(x-y)\]
the greenn function satisfy the equation:
\[(i\slashed{\partial}_x-m)S_R(x-y)=i\delta^4(x-y)I\]
the fourier transform of  the retarded green function is:
\[S_R(p)=\frac{i(\slashed{p}+m)}{p^2-m^2}\]
the feymann propagator:
\[S_F(x-y)=\int \frac{d^4p}{(2\pi)^4}\frac{i(\slashed{p}+m)}{p^2-m^2+i\epsilon}e^{-i(x-y)}\]
\[S_F(p)=\frac{i(\slashed{p}+m)}{p^2-m^2+i\epsilon}\]

\subsection{discrete symmetrics in dirac field}
\textcolor{blue}{parity P,time reversal T,and chage interchage C}\par
1.Parity P:reverse the momentum but preserve the spin:
\[a_p^{s^\dagger}\diracr{0}\stackrel{P}{\longrightarrow}a_{-p}^{s^\dagger}\diracr{0}\]
\[Pa_p^sP=\eta_a a_{-p}^s\]
\[Pb_p^sP=\eta_b b_{-p}^s\]
\[\eta_a\eta_b=-1\]
\[P\Psi(t,x)P=\eta_a\gamma^0\Psi(t,-x)\]
\[P\bar{\Psi(t,x)}P=\eta_a^*\bar{\Psi(t,-x)}\gamma^0\]
2.Time Reversal T:reverse the momentum and spin\par
time reversal operator also act on the c-number:
\[T(c)=c^*T\]
define two vector operator:
\[a_p^s==(a_p^2,-a_p^1),b_p^s==(b_p^2,-b_p^1)\]
then time reversal operator T has the property:
\[Ta_p^sT=a_{-p}^{-s},Tb_p^sT=b_{-p}^{-s}\]
\[T\Psi(t,x)T=-\gamma^1\gamma^3\Psi(-t,x)\]
\[T\bar{\Psi}(t,x)T=\bar{\Psi}(-t,x)\gamma^1\gamma^3\]

3.charge conjugation C\par
\[Ca_p^sC=b_p^s,Cb_p^sC=a_p^s\]
\[C\Psi(x)C=-i(\bar{\Psi}\gamma^0\gamma^2)^{T}\]
\[C\bar{\Psi}C=(-i\gamma^0\gamma^2\Psi)^{T}\]

