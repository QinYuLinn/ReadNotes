\subsection{平移群的不可约表示与布洛赫定理}
正格矢量($a_i$为基矢):
\[R_l=\sum_{i=1}^3l_i a_i\]
倒易空间的基矢为:
\[b_1=\frac{2\pi}{\Omega}a_2\times a_3\]
\[b_2=\frac{2\pi}{\Omega}a_3\times a_1\]
\[b_3=\frac{2\pi}{\Omega}a_1\times a_1\]
倒易空间的格点矢量为:
\[K_l=\sum_{i=1}^3l_i b_i\]
当晶体里面有N($N_1N_2N_3$)个元胞时,系统的平移群有N个元素,由于平移群每个元素自成一个共轭类,因此这个平移群有N个不可约表示,因而他有N个1维的不可约表示:
\[k=\sum_{j=1}^3\frac{n_j}{N_j}b_j\]
由于在第一布里渊区内$n_j$有$N_j$种可能,因此共有N个k,每个k可以作为不可约表示的标记,记相应的一维表示的基函数为$\Psi_k(r)$,那么有布洛赫定理:
\[\Psi_k(r)=e^{ik\cdot r}u_k(r),u_k(r+R_l)=u_k(r)\]
从平移对称性出发,可以得到几个常用的等式:
\[\sum_{R_l}e^{i(k-k')\cdot R_l}=N\delta_{k,k'}\]
\[\sum_{k\in BZ}e^{-ik\cdot (R_l-R_s)}=N\delta_{R_l R_s}\]
\[\sum_{k}\rightarrow \frac{1}{(2\pi)^3}\int d^3k\]
当系统具有平移对称性时,T与H对易,因而有相同的本征函数,对于一个单电子来讲,则有:
\[H\Psi_k(r)=E(k)\Psi_k(r)\]
\[H_ku_k(r)=E(k)u_k(r)\]
由于$u_k(r)$具有平移对称性,因此只需要在单个元胞内求解薛定谔方程,因而对于其能级是分立的,对于一个k有:
\[E_1(k),E_2(k),\cdots\]
相应的有波函数:
\[\u_{nk}(r)\]
也就是说相应的有
\[H\Psi_{nk}(r)=E_n(k)\Psi_{nk}(r)\]
由于:
\[\Psi_{n,k}=\Psi_{n,k+K_n}\]
从而有:
\[E_{n}(k)=E_{n}(k+K_n)\]
从而能区是周期性的,只需要考虑第一布里渊区即可。\par
因此取定n之后,能量由于周期性,必有上下界,因而同一个n不同的k能量是一个带,因此存在能带。不同的n对应于不同的带,整体成为晶体的带结构。\par
\subsection{晶体对称性及其对能带的影响}
空间群:
\[\{\alpha|t\}\]
其中$\alpha$代表旋转,t代表平移$\alpha =E$(E代表恒等变换)时代表平移群,$r=0$时代表点群;
\[\{\alpha|t\}r=\alpha r+t\]
晶体的空间群定义为:空间群的子群使得平移群作为其不变子群。\par
由于
\[\spacegroup{\alpha}{t}\spacegroup{E}{R_l}\spacegroup{\alpha}{t}^{-1}=\spacegroup{E}{\alpha R_l}\}\]
不变子群的条件要求$\alpha R_l$仍是正格矢量,因此旋转轴只可能是2,3,4,6次,从而晶体的空间群必定是有限群。\par
假设晶体的空间群中有元素$\spacegroup{\beta}{t}$,那么可以证明:
\[E_n(\beta k)=E_n(k)\]
因此如果某个空间群元素保持k(=k或者$k+K_n$)不变,那么能级$E_n(k)$可能简并的,因此定义保持k不变的点群称为k波矢群。
\[\alpha k=k+K_n\]
k波矢群不可约表示的维数即为$E_n(k)$的简并度。\par
由布里渊区的定义可知,点阵的全部点群对称性必然与BZ的点群对称性相同。
