\subsection{贝塞尔函数}
\begin{enumerate}
\item{$\nu$阶贝塞尔微分算子\par}
\[B_\nu=\frac{1}{x}\frac{d}{dx}x\frac{d}{dx}+(1-\frac{\nu^2}{x^2})\]
其线性无关的两组解为(n不是整数):
\[J_\nu,J_{-\nu}\]
其线性无关的两组解为(n是整数):
\[J_\nu,N_{\nu}\]
其中我们有:
\[N_{\nu}(x)=\frac{\cos \nu \pi J_\nu(x)-J_{-\nu}(x)}{\sin\nu\pi}\]
贝塞尔函数的渐近形式为:
\[J_\nu(x)\sim \sqrt{\frac{2}{\pi x}}\cos(x-\frac{\nu\pi}{2}-\frac{\pi}{4})\]
\[N_\nu(x)\sim \sqrt{\frac{2}{\pi x}}\sin(x-\frac{\nu\pi}{2}-\frac{\pi}{4})\]
既包含汇聚波又包含发散波,因此定义hankel函数将汇聚与发散波分离:
\[H_\nu^{(1)}(x)=J_\nu(x)+iN_\nu(x)\]
\[H_\nu^{(2)}(x)=J_\nu(x)-iN_\nu(x)\]
分别代表发散播与汇聚波\par\vspace{10pt}
\item{$\nu$阶虚宗量贝塞尔微分算子}
\[B^i_\nu=\frac{1}{x}\frac{d}{dx}x\frac{d}{dx}+(-1-\frac{\nu^2}{x^2})\]
在此方程中令$t=ix$,那么微分算子就变为$\nu$阶贝塞尔微分算子。
虚宗量贝塞尔微分算子的解为贝塞尔函数:
\begin{align*}
I_\nu(x)&=e^{-\frac{i\nu\pi}{2}}J_\nu(xi)\\
K_\nu(x)&=\frac{\pi}{2\sin\nu\pi}(I_{-\nu}(x)-I_\nu(x))
\end{align*}
在$x\rightarrow \infty$时的渐进行为为:
\begin{align*}
I_\nu(x)&\sim \sqrt{\frac{1}{2\pi x}}e^x\\
K_\nu(x)&\sim \sqrt{\frac{\pi}{2 x}}e^{-x}
\end{align*}
\item{$\nu$阶球贝塞尔微分算子}
\[B^s_\nu=\frac{1}{x^2}\frac{d}{dx}x^2\frac{d}{dx}+(1-\frac{\nu(\nu+1)}{x^2})\]
可以对待求解的函数做变换:
\[y(x)=\frac{v(x)}{\sqrt{x}}\]
那么关于v(x)的方程就是$\nu+\frac{1}{2}$阶贝塞尔微分算子满足的方程:
\[B_{\nu+\frac{1}{2}}(x)v(x)=0\]
因此$\nu$阶球贝塞尔微分算子的解为球贝塞尔函数:
\[j_\nu(x)=\sqrt{\frac{\pi}{2x}}J_{\nu+\frac{1}{2}}(x)\]
\[n_\nu(x)=\sqrt{\frac{\pi}{2x}}N_{\nu+\frac{1}{2}}(x)\]
同样可以因此定义球hankel函数将汇聚与发散波分离:
\[h_\nu^{(1)}(x)=j_\nu(x)+in_\nu(x)\]
\[h_\nu^{(2)}(x)=j_\nu(x)-in_\nu(x)\]
\end{enumerate}

\subsection{球函数}
\begin{enumerate}
\item{连带勒让德微分算子$L_l^m$\par}
\[L_l^m=\frac{d}{dx}[(1-x^2)\frac{d}{dx}]+(l(l+1)-\frac{m^2}{1-x^2})\]
该微分算子可以看作是微分算子:
\[L_l^m=\frac{1}{\sin\theta}\frac{d}{d\theta}\sin\theta\frac{d}{d\theta}+[l(l+1)-\frac{m^2}{\sin^2\theta}]\]
做变量替换$x=\cos\theta$而得到的.
该方程的解为连带勒让德函数:
\[P_l^m(x):=P_l^m(\cos\theta)\]
\end{enumerate}
