\begin{equation}
V_n=\int_{\sum x_i^2\leq R^2} \prod dx_1\cdots dx_n
\end{equation}
先从数学的角度严格计算这个积分,可以做n维的球坐标变换
\begin{align*}
&x_1=r\sin\theta_1\sin\theta_2\cdots \sin\theta_{n-2}\sin\theta_{n-1}\\
&x_2=r\sin\theta_1\sin\theta_2\cdots \sin\theta_{n-2}\cos\theta_{n-1}\\
&x_3=r\sin\theta_1\sin\theta_2\cdots \cos\theta_{n-2}\\
&\cdots\\
&x_{n-1}=r\sin\theta_1\cos\theta_2\\
&x_{n}=r\cos\theta_1
\end{align*}
利用归纳发计算上述变换的雅可比行列式为:
\[\prod dx_1\cdots dx_n=r^{n-1}\sin^{n-2}\theta_1\sin^{n-3}\theta_2\cdots \sin\theta_{n-2} drd\theta_1 d\theta_2\cdots d\theta_{n-1}\]
利用基本的积分:
\begin{align*}
&\int_{0}^{\frac{\pi}{2}} \sin^n\theta d\theta=\frac{(n-1)!!}{n!!}\frac{\pi}{2}, n-even\\
&\int_{0}^{\frac{\pi}{2}} \sin^n\theta d\theta=\frac{(n-1)!!}{n!!}, n-odd\\
\end{align*}
从而可以得到积分结果为:
\begin{align*}
&V_n=R^n\frac{1}{n!!}2^{n-2}(\frac{\pi}{2})^{\frac{n-2}{2}} 2\pi , n-even\\
&V_n=R^n\frac{1}{n!!}2^{n-2}(\frac{\pi}{2})^{\frac{n-3}{2}} 2\pi , n-odd\\
\end{align*}
利用gamma函数可将上面的式子统一写为:
\[V_n=\frac{\pi^{\frac{n}{2}}}{\Gamma(\frac{n}{2}+1)}R^n\]
由于:
\[dV_n=S_{n}dr\]
从而可以得到
\[S_{n}=\int_{\theta_j}r^{n-1}\sin^{n-2}\theta_1\sin^{n-3}\theta_2\cdots \sin\theta_{n-2} d\theta_1 d\theta_2\cdots d\theta_{n-1}=\frac{2\pi^{\frac{n}{2}}}{\Gamma(\frac{n}{2})}r^{n-1}\]
对于物理学家来说,没有必要做这么详细的严格的推导,一般地,我们有:
\[V_n(R)=C_nR^n\]
从而有
\[S_n(R)=nC_nR^{n-1}\]
另一方面:
\[\pi^{\frac{n}{2}}=(\int_{-\infty}^{\infty}e^{-x^2}dx)^n=\int\int\cdots\int e^{-(x_1^2+\cdots x_n^2)}dV_n\]
从而有:
\[\pi^{\frac{n}{2}}=\int_0^\infty e^{-r^2}nC_n r^{n-2}=C_n\Gamma(\frac{n}{2}+1)\]
进而确定这个常数为:
\[C_n=\frac{\pi^{\frac{n}{2}}}{\Gamma(\frac{n}{2}+1)}\]
这种类型的积分也可以利用归纳法求解,例如我们考虑如下的积分:
\[S(n,R)=\int_{\sum_{1}^n |r_i|<R}d^3r_1\cdots d^3r_n\]
计算S(n+1,R),我们可以得到:
\begin{align*}
S(n+1,R)&=\int_{\sum_{1}^{n+1} |r_i|<R}d^3r_1\cdots d^3r_nd^3r_{n+1}\\
&=\int_0^Rd^3r_{n+1} \int_{\sum_{1}^n |r_i|<R-|r_{n+1}|}d^3r_1\cdots d^3r_n
\end{align*}
n=1时结果为:
\[S(1,R)=\int_{|r_1|<R}d^3r=\int_0^R 4\pi r^2=\frac{4\pi}{3}R^3\]
在计算S(2,R)找规律,最后归纳假设得到结果:
\[S(n,R)=\frac{(8\pi)^N}{(3N)!}R^{3N}\]
